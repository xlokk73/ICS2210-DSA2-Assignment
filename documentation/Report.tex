\documentclass[a4paper, 12pt]{article}
\usepackage{graphicx}

\setlength\parindent{24pt}

\begin{document}

\begin{figure}
    \centering
    \includegraphics[width=1\textwidth]{Logo}
\end{figure}

\title{Assignment Report}
\author{Manwel Bugeja}
\date{\today}
\maketitle

\tableofcontents
\newpage

\section{Introduction}
\subsection{Note on the code}
This assignment is implemented in c++. Readability was prioritized over efficiency since the code needs to be
easily understood by others. 

\section{Part 1}
\subsection{How the problem was tackled}
\subsubsection{Structures}
First off, the needed structures where created: 'variable' in an enumerated type containing the possible variables.
An 'invalid' variable was also created whose use is for error catching purposes. Following that, a structure 'literal\_t'.
This structure is composed of a variable and boolean to show whether the variable is negated or not.

Then a clause was defined as a vector of literals. Similarly, a formula was defined as a vector of clauses.

Apart from those, 'expression\_t' was also defined as an array of alphabet. Alphabet being an enumeration
containing the possible alphabet characters received as input.

These structures are defined in 'sat.h'.

\subsubsection{Parsing}
In the parsing, the string is first converted to an 'expression\_t'. Then it is translated to a formula. Since all variables are 
only a character long, the commas can be ignored completly when the expression is inputted. For example "(wx), (!w)" will still be parsed successfully. 
This will not reslut in an error as the input can still be successfully parse. 
Still, inserting "(," will still cause the program to exit since after an open parethesis, 
the parser expects a literal (variable or negation followed by a variable).

\subsection{The DPLL algorithm}
The DPLL algorithm was implemented according to the pseudo code listed in the course notes. In the 'choose\_literal()' part of the algorithm,
the first literal from the left is chosen.

\subsection{Testing}
For the testing of the algorithm, some expressions where tried, with the output compared.


\section{Part 2}

\end{document}
